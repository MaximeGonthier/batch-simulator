\documentclass[a4paper]{article}
\usepackage[utf8]{inputenc}
\usepackage{color}
\usepackage[dvipsnames]{xcolor}
\usepackage{graphicx}
\usepackage{amssymb}
\usepackage{amsfonts}
\usepackage{diagbox}
\usepackage{colortbl}
\usepackage{pdfpages}
\usepackage{todonotes}
\usepackage{listings}
\usepackage{amsmath}
\usepackage{caption}
\usepackage{subcaption}
\usepackage{xspace}
\usepackage{xcolor,pifont}
\usepackage{fullpage}
\usepackage{algorithm, algpseudocode}
\usepackage[hidelinks]{hyperref}
\usepackage[english]{babel}
%~ \usepackage[backend=bibtex]{biblatex}
%~ \bibliography{ref_cadre}
\algnewcommand\algorithmicforeach{\textbf{for each}}
\algdef{S}[FOR]{ForEach}[1]{\algorithmicforeach\ #1\ \algorithmicdo}
\renewcommand{\algorithmicrequire}{\textbf{Input:}}
\renewcommand{\algorithmicensure}{\textbf{Output:}}
\newtheorem{Problem}{Problem}
\newtheorem{Theorem}{Theorem}
\newtheorem{Lemma}{Lemma}
\newtheorem{Hypothesis}{Hypothesis}
\providecommand{\keywords}[1]{\textbf{\textit{Key-words:}} #1}
\newcommand{\LM}[1]{\textcolor{red}{LM:~{#1}}}
\newcommand{\ST}[1]{\textcolor{orange}{ST:~{#1}}}
\newcommand{\TODO}[1]{\textcolor{olive}{TODO:~{#1}}}
\newcommand{\Q}[1]{\textcolor{blue}{Q:~{#1}}}
\setlength{\parskip}{0.2 cm}
\title{Notes articles}
\author{Maxime GONTHIER Samuel THIBAULT Loris MARCHAL}

\newcommand{\card}[1]{\ensuremath{\left|{#1}\right|}\xspace}
\newcommand{\live}{\ensuremath{L}\xspace}
\newcommand{\evict}{\ensuremath{\mathcal{V}}\xspace}
\newcommand{\dataset}{\ensuremath{\mathbb{D}}\xspace}
\newcommand{\taskset}{\ensuremath{\mathbb{T}}\xspace}
\newcommand{\packageset}{\ensuremath{\mathbb{P}}\xspace}
\newcommand{\inputs}{\ensuremath{\mathcal{D}}\xspace}
\newcommand{\optpb}{\textsc{Min\-Loads\-For\-Tasks\-Sharing\-Data}\xspace}
\newcommand{\nbloads}{\ensuremath{\mathit{\mathit{\#Loads}}}\xspace}
\newcommand{\MIN}{\ensuremath{\mathit{MIN}}\xspace}
\newcommand{\starpu}{\textsc{StarPU}\xspace}
\newcommand{\nbPU}{\ensuremath{\vert PU \vert}\xspace}

\usepackage{tikz}
\usetikzlibrary{fit, shapes.geometric, patterns}

\makeatletter\tikzset{hatch distance/.store in=\hatchdistance,hatch distance=5pt,hatch thickness/.store in=\hatchthickness,hatch thickness=5pt}\pgfdeclarepatternformonly[\hatchdistance,\hatchthickness]{north east hatch}{\pgfqpoint{-1pt}{-1pt}}{\pgfqpoint{\hatchdistance}{\hatchdistance}}{\pgfpoint{\hatchdistance-1pt}{\hatchdistance-1pt}}{\pgfsetcolor{\tikz@pattern@color}\pgfsetlinewidth{\hatchthickness}\pgfpathmoveto{\pgfqpoint{0pt}{0pt}}\pgfpathlineto{\pgfqpoint{\hatchdistance}{\hatchdistance}}\pgfusepath{stroke}}\makeatother\usetikzlibrary{calc,shadings,patterns,tikzmark}\newcommand\HatchedCell[5][0pt]{\begin{tikzpicture}[overlay,remember picture]\path ($(pic cs:#2)!0.5!(pic cs:#3)$)coordinate(aux1)(pic cs:#4)coordinate(aux2);\fill[#5]($(aux1)+(-0.23*0.075\textwidth,1.9ex)$)rectangle($(aux1 |- aux2)+(0.23*0.075\textwidth,-#1*\baselineskip-.8ex)$);\end{tikzpicture}}

\newcounter{nodemarkers}
\newcommand\circletext[1]{%
    \tikz[overlay,remember picture] 
        \node (marker-\arabic{nodemarkers}-a) at (0,1.5ex) {};%
    #1%
    \tikz[overlay,remember picture]
        \node (marker-\arabic{nodemarkers}-b) at (0,0){};%
    \tikz[overlay,remember picture,inner sep=2pt]
        \node[draw,ellipse,fit=(marker-\arabic{nodemarkers}-a.center) (marker-\arabic{nodemarkers}-b.center)] {};%
    \stepcounter{nodemarkers}%
}

% Pour les checkmark
\newcommand*\colourcheck[1]{%
  \expandafter\newcommand\csname #1check\endcsname{\textcolor{#1}{\ding{52}}}
}
\colourcheck{green}
\newcommand{\ok}[1]{\textcolor{ForestGreen}{OK \greencheck}}
\newcommand{\oktext}[1]{\textcolor{ForestGreen}{OK \greencheck}~\textcolor{ForestGreen}{#1}}

\begin{document}

%~ \tableofcontents
\newpage

\section{What's new since last meeting ?}

	\begin{enumerate}
		\item Diapos
		\item Correction de maximum use single file
		\item Code to have 3 days and keep only the second one for stats
	\end{enumerate}
	
\section{Questions for next meeting}

	\begin{enumerate}
		\item It's now a 100 seconds frame to have similar input files for 2 jobs of the same user. Is it ok ?
		\item Stefanos
		\item Plénière: Prix inscription et présentation
		\item Change percentage of task with 256 or 1024 files? 
	\end{enumerate}

\section{Todo}
	\subsection{General}
		\begin{enumerate}
			\item Similar results: look if it's exactly the same or it is because the cluster is mainly empty ? Do they take exactly the same decision ? Does easy back fill really change anything or it just reschedule like FCFS ?
			\item Elongation: multiply each submission time by a constant
			\item THe cluster is empty most of the time: Test with a smaller cluster but not 4 nodes.
			\item THe cluster is empty most of the time: Test with 3 phases and full cluster
			\item Idea algo: 6.2 + penalty for fcfs + easybf or conservative bf
			\item Code something to use a workload with 3 phase. Example: Day 1 2 and 3. Use the same scheduler for all 3 days. Only jobs in day 2 files are taken into consideration in the results.
			\item Bonus in depth slide about eviction for Stefano
			\item Se ré inscrire à l'ENS avant le 15 Juin
			\item Refelchir a maximum use en mieux
			\item Agreger plusieurs jours d'affiler et enlever debt et fin pour voir un cas lancé
			\item faire des boxplots
			\item faire option easy bf a la place de shceuler en particulier
			\item surprr code inutile
			\item Voir papiers Stefanos
			\item Voir thèses de Herman
			\item Talk to Hans Karlsson
			\item Talk to David Black-Schaffer
			\item Lire articles Emanuel Rubensson \url{https://webmail.ens-lyon.fr/?_task=mail&_caps=pdf%3D1%2Cflash%3D0%2Ctif%3D0&_uid=4630&_mbox=INBOX&_action=show}
			\item Compare algo with different workloads. Use queue time of each job compared to a baseline. Compare queue time of each job between 2 heuristics by saying who won on each queue time.
			\item Ajouter algos qui reschedule de temps en temps
			\item Code to compare ourselves, HEFT that would be FCFS that take into consideration transfer time for the earliest available cores
			\item Use same X scale for distributions of queue times
			\item Use same X scale workloads stats
			\item Use same X scale for some gantt charts when I want to compare them
			\item Code algo 2
			\item vidéo IPDPS avant le 15 mai
			\item TODO du code: backfill, available node list qui contient les cores available aussi
			\item contrainte de la localié sur la taille de la données, contrainte sur les coeurs, contrainte sur les partage de données
			\item stratégie qui essaye d'utiliser moins les gros noeud pour les garder pour les gros jobs ?
			\item Lors d'une exec imprimer sur le terminal et dans le fichier de résultats les stats sur le workload et le cluster
			\item Comparer algo qui reschedule tout et algo qui schedule que quand un nouveau core est disponible et ne schedule que 1 seul job a la fois par cores.
			\item lire article bf : Utilization, Predictability, Workloads
			\item Faire plusieurs test et mettre une barre d'erreur
			\item Afficher temps de transferts sur les Gantt charts
		\end{enumerate}
	\subsection{Batsim}
		\begin{enumerate}
			\item Batsim a pas la granularité au sein d'un noeud
			\item Gérer n nodes
			\item Gérer n jobs
			\item Faire un delay aussi long que la somme du poids des données manquantes
			\item faire la maj des données du node partout, sois Dans le scheduler sois dans fit mais faut le faire!
			\item Gérér à la main les évictions
		\end{enumerate}

\end{document}
