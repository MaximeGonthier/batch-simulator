\documentclass[a4paper]{article}
\usepackage[utf8]{inputenc}
\usepackage{color}
\usepackage[dvipsnames]{xcolor}
\usepackage{graphicx}
\usepackage{amssymb}
\usepackage{amsfonts}
\usepackage{diagbox}
\usepackage{colortbl}
\usepackage{pdfpages}
\usepackage{todonotes}
\usepackage{listings}
\usepackage{amsmath}
\usepackage{caption}
\usepackage{subcaption}
\usepackage{xspace}
\usepackage{xcolor,pifont}
\usepackage{fullpage}
\usepackage{algorithm, algpseudocode}
\usepackage[hidelinks]{hyperref}
\usepackage[english]{babel}
%~ \usepackage[backend=bibtex]{biblatex}
%~ \bibliography{ref_cadre}
\algnewcommand\algorithmicforeach{\textbf{for each}}
\algdef{S}[FOR]{ForEach}[1]{\algorithmicforeach\ #1\ \algorithmicdo}
\renewcommand{\algorithmicrequire}{\textbf{Input:}}
\renewcommand{\algorithmicensure}{\textbf{Output:}}
\newtheorem{Problem}{Problem}
\newtheorem{Theorem}{Theorem}
\newtheorem{Lemma}{Lemma}
\newtheorem{Hypothesis}{Hypothesis}
\providecommand{\keywords}[1]{\textbf{\textit{Key-words:}} #1}
\newcommand{\LM}[1]{\textcolor{red}{LM:~{#1}}}
\newcommand{\ST}[1]{\textcolor{orange}{ST:~{#1}}}
\newcommand{\TODO}[1]{\textcolor{olive}{TODO:~{#1}}}
\newcommand{\Q}[1]{\textcolor{blue}{Q:~{#1}}}
\setlength{\parskip}{0.2 cm}
\title{Notes articles}
\author{Maxime GONTHIER Samuel THIBAULT Loris MARCHAL}

\newcommand{\card}[1]{\ensuremath{\left|{#1}\right|}\xspace}
\newcommand{\live}{\ensuremath{L}\xspace}
\newcommand{\evict}{\ensuremath{\mathcal{V}}\xspace}
\newcommand{\dataset}{\ensuremath{\mathbb{D}}\xspace}
\newcommand{\taskset}{\ensuremath{\mathbb{T}}\xspace}
\newcommand{\packageset}{\ensuremath{\mathbb{P}}\xspace}
\newcommand{\inputs}{\ensuremath{\mathcal{D}}\xspace}
\newcommand{\optpb}{\textsc{Min\-Loads\-For\-Tasks\-Sharing\-Data}\xspace}
\newcommand{\nbloads}{\ensuremath{\mathit{\mathit{\#Loads}}}\xspace}
\newcommand{\MIN}{\ensuremath{\mathit{MIN}}\xspace}
\newcommand{\starpu}{\textsc{StarPU}\xspace}
\newcommand{\nbPU}{\ensuremath{\vert PU \vert}\xspace}

\usepackage{tikz}
\usetikzlibrary{fit, shapes.geometric, patterns}

\makeatletter\tikzset{hatch distance/.store in=\hatchdistance,hatch distance=5pt,hatch thickness/.store in=\hatchthickness,hatch thickness=5pt}\pgfdeclarepatternformonly[\hatchdistance,\hatchthickness]{north east hatch}{\pgfqpoint{-1pt}{-1pt}}{\pgfqpoint{\hatchdistance}{\hatchdistance}}{\pgfpoint{\hatchdistance-1pt}{\hatchdistance-1pt}}{\pgfsetcolor{\tikz@pattern@color}\pgfsetlinewidth{\hatchthickness}\pgfpathmoveto{\pgfqpoint{0pt}{0pt}}\pgfpathlineto{\pgfqpoint{\hatchdistance}{\hatchdistance}}\pgfusepath{stroke}}\makeatother\usetikzlibrary{calc,shadings,patterns,tikzmark}\newcommand\HatchedCell[5][0pt]{\begin{tikzpicture}[overlay,remember picture]\path ($(pic cs:#2)!0.5!(pic cs:#3)$)coordinate(aux1)(pic cs:#4)coordinate(aux2);\fill[#5]($(aux1)+(-0.23*0.075\textwidth,1.9ex)$)rectangle($(aux1 |- aux2)+(0.23*0.075\textwidth,-#1*\baselineskip-.8ex)$);\end{tikzpicture}}

\newcounter{nodemarkers}
\newcommand\circletext[1]{%
    \tikz[overlay,remember picture] 
        \node (marker-\arabic{nodemarkers}-a) at (0,1.5ex) {};%
    #1%
    \tikz[overlay,remember picture]
        \node (marker-\arabic{nodemarkers}-b) at (0,0){};%
    \tikz[overlay,remember picture,inner sep=2pt]
        \node[draw,ellipse,fit=(marker-\arabic{nodemarkers}-a.center) (marker-\arabic{nodemarkers}-b.center)] {};%
    \stepcounter{nodemarkers}%
}

% Pour les checkmark
\newcommand*\colourcheck[1]{%
  \expandafter\newcommand\csname #1check\endcsname{\textcolor{#1}{\ding{52}}}
}
\colourcheck{green}
\newcommand{\ok}[1]{\textcolor{ForestGreen}{OK \greencheck}}
\newcommand{\oktext}[1]{\textcolor{ForestGreen}{OK \greencheck}~\textcolor{ForestGreen}{#1}}

\begin{document}

%~ \tableofcontents
\newpage

%~ /data-aware-batch-scheduling/MBSS$ oarsub -p orion -l core=2,walltime=08:00:00 -r '2022-06-14 19:00:00' "bash Stats_single_workload.sh inputs/workloads/converted/test-11 inputs/clusters/rackham_450_128_32_256_4_1024.txt Fcfs 0"

\section{What's new since last meeting ?}

	\begin{enumerate}
		\item Re-writing of area filling
		\item More heatmaps fcfs with a score and all multipliers combination heatmaps
		\item Code area filling ratio and area filling omniscient
		\item pseudo code area filling with different filling of allocated area (It's: Allocated Area[choosen size][x] gets Allocated Area[choosen size][x] + Area(Ji) with Area(Ji) = cores*walltime. But I would prefer to do: cores*(delay+transfer time))
		\item code area filling with ratio and allocated area
		\item code easybf fcfs only
		\item are filling omnisicnet comme je fais en ce moment. Si ca marche pas essayer avec le graduellement
		\item dans area filling: A la place de immedialy est ce que ca finir a plus tot sur le noeud de taille x + 1 et comparer aussi avec x + 2 ... x + n.
		\item Size constraint: sorting by size is bad if normal cluster
		\item ignore dt for constraint == 2
		\item revoir area filling tart dans le main en fct de planned or ratio et des nouveau noms de fichiers et retester avec les 2 cluster différents.
		\item plot with nb of upgraded jobs
		\item plot avec backfill. Ameliore fcfs mais fcfs score est meilleur sans. La diff e transfert est grande entre fcfs score et fcfs score bf
		\item Why are filling so bad ? Even in omniscient ? I corrected a bit but it's still worse. Also there is a difference between true allocated area and the one I compute in the schedule (because the scheduele change and I only update allocated area when a job start. Else I have a temp one just for the current schedule).
		\item Try in area filling to reduce allocated area by the diff between walltime and real delay if the job finished before it's walltime
		\item Comment one paper of vincenc
		\item Mail au gars de data locality sur numa: Jannis Klinkenberg
		\item dossier inscription administrative
		\item attestation sur siged de acaces et IPDPS
		\item plot the number of jobs where the queue time is superior to 25000 for example
		\item tirer quelques conclusions dans le document principal
	\end{enumerate}
	
\section{Questions for next meeting}

	\begin{enumerate}
		\item Do I still put a data on each job ?
		\item Utiliser attestation volontariat IPDPS pour formations ? 20h ? Pareil pour ACACES avec 20h de cours ?
		\item Dois-je annuler ma demande de cumul ?
		\item Still, no corelation between node size and job length. Should I add one ?
	\end{enumerate}

\section{Todo}
	\subsection{General}
		\begin{enumerate}
			\item check que ca marche le area filling de fcfs with a score
			\item refaire test size constraint a cause de start jobs
			\item test area fillling avec et sans ratio sur workload un peu plus grand pour voir.
			\item un score sur la categorie de node utilisé (pénalité si plus haut ?) + mode 1 avec queue time moyen actuel. Fusionner fcfs score et area et/ou backfill big nodes mode 1 et mettre la contrainte des tailles et les temps de chargements. Tester aussi en triant
			\item Courbe en fixant 1 paramèrte pour ci dessus ?
			\item transmettre dossier siged  et payer une fois tous les avis saisies (5/6 ou 6/6)
			\item paye 3 missions + enseirb
			\item mail mathieu calendrier: attendre réponse
			\item Répondre a Bora
			\item generation de workload : aller chercher loin puis cut le fichier plus tôt pour pas avoir trop de jobs non plus.
			\item Check rendu rapports le 30 Aout pour les 3A
			\item Date redu rapport des 2A ?
			\item ajouter easy bf a fcfs avec un score						
			\item tracer terminaison des jobs -2 sur courbes usage du cluster
			\item Ne pas oublier que j'ai les courbes VS sur flow stretch et queue times aussi que je peux tracer						
			\item Code to compare ourselves, HEFT that would be FCFS that take into consideration transfer time for the earliest available cores
			\item Use same X scale for workloads stats when I want to compare them
			\item Use same X scale for some gantt charts when I want to compare them
			\item lire article bf : Utilization, Predictability, Workloads et le mettre dans les related works: It gives information about the imporvement given by backfilling. A complex conservative (loking at job size) backfilling can be better than easy, but it still depend on the workload.
			\item Faire plusieurs test et mettre une barre d'erreur
			\item Afficher temps de transferts sur les Gantt charts: Comment ?
			\item Rendre le code général (20 pour cores par node et 3 pour nb tailles de noeuds différents)
		\end{enumerate}
		
	%~ \subsection{Batsim}
		%~ \begin{enumerate}
			%~ \item Batsim a pas la granularité au sein d'un noeud
			%~ \item Gérer n nodes
			%~ \item Gérer n jobs
			%~ \item Faire un delay aussi long que la somme du poids des données manquantes
			%~ \item faire la maj des données du node partout, sois Dans le scheduler sois dans fit mais faut le faire!
			%~ \item Gérér à la main les évictions
		%~ \end{enumerate}

\end{document}
