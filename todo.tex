\documentclass[a4paper]{article}
\usepackage[utf8]{inputenc}
\usepackage{color}
\usepackage[dvipsnames]{xcolor}
\usepackage{graphicx}
\usepackage{amssymb}
\usepackage{amsfonts}
\usepackage{diagbox}
\usepackage{colortbl}
\usepackage{pdfpages}
\usepackage{todonotes}
\usepackage{listings}
\usepackage{amsmath}
\usepackage{caption}
\usepackage{subcaption}
\usepackage{xspace}
\usepackage{xcolor,pifont}
\usepackage{fullpage}
\usepackage{algorithm, algpseudocode}
\usepackage[hidelinks]{hyperref}
\usepackage[english]{babel}
%~ \usepackage[backend=bibtex]{biblatex}
%~ \bibliography{ref_cadre}
\algnewcommand\algorithmicforeach{\textbf{for each}}
\algdef{S}[FOR]{ForEach}[1]{\algorithmicforeach\ #1\ \algorithmicdo}
\renewcommand{\algorithmicrequire}{\textbf{Input:}}
\renewcommand{\algorithmicensure}{\textbf{Output:}}
\newtheorem{Problem}{Problem}
\newtheorem{Theorem}{Theorem}
\newtheorem{Lemma}{Lemma}
\newtheorem{Hypothesis}{Hypothesis}
\providecommand{\keywords}[1]{\textbf{\textit{Key-words:}} #1}
\newcommand{\LM}[1]{\textcolor{red}{LM:~{#1}}}
\newcommand{\ST}[1]{\textcolor{orange}{ST:~{#1}}}
\newcommand{\TODO}[1]{\textcolor{olive}{TODO:~{#1}}}
\newcommand{\Q}[1]{\textcolor{blue}{Q:~{#1}}}
\setlength{\parskip}{0.2 cm}
\title{Notes articles}
\author{Maxime GONTHIER Samuel THIBAULT Loris MARCHAL}

\newcommand{\card}[1]{\ensuremath{\left|{#1}\right|}\xspace}
\newcommand{\live}{\ensuremath{L}\xspace}
\newcommand{\evict}{\ensuremath{\mathcal{V}}\xspace}
\newcommand{\dataset}{\ensuremath{\mathbb{D}}\xspace}
\newcommand{\taskset}{\ensuremath{\mathbb{T}}\xspace}
\newcommand{\packageset}{\ensuremath{\mathbb{P}}\xspace}
\newcommand{\inputs}{\ensuremath{\mathcal{D}}\xspace}
\newcommand{\optpb}{\textsc{Min\-Loads\-For\-Tasks\-Sharing\-Data}\xspace}
\newcommand{\nbloads}{\ensuremath{\mathit{\mathit{\#Loads}}}\xspace}
\newcommand{\MIN}{\ensuremath{\mathit{MIN}}\xspace}
\newcommand{\starpu}{\textsc{StarPU}\xspace}
\newcommand{\nbPU}{\ensuremath{\vert PU \vert}\xspace}

\usepackage{tikz}
\usetikzlibrary{fit, shapes.geometric, patterns}

\makeatletter\tikzset{hatch distance/.store in=\hatchdistance,hatch distance=5pt,hatch thickness/.store in=\hatchthickness,hatch thickness=5pt}\pgfdeclarepatternformonly[\hatchdistance,\hatchthickness]{north east hatch}{\pgfqpoint{-1pt}{-1pt}}{\pgfqpoint{\hatchdistance}{\hatchdistance}}{\pgfpoint{\hatchdistance-1pt}{\hatchdistance-1pt}}{\pgfsetcolor{\tikz@pattern@color}\pgfsetlinewidth{\hatchthickness}\pgfpathmoveto{\pgfqpoint{0pt}{0pt}}\pgfpathlineto{\pgfqpoint{\hatchdistance}{\hatchdistance}}\pgfusepath{stroke}}\makeatother\usetikzlibrary{calc,shadings,patterns,tikzmark}\newcommand\HatchedCell[5][0pt]{\begin{tikzpicture}[overlay,remember picture]\path ($(pic cs:#2)!0.5!(pic cs:#3)$)coordinate(aux1)(pic cs:#4)coordinate(aux2);\fill[#5]($(aux1)+(-0.23*0.075\textwidth,1.9ex)$)rectangle($(aux1 |- aux2)+(0.23*0.075\textwidth,-#1*\baselineskip-.8ex)$);\end{tikzpicture}}

\newcounter{nodemarkers}
\newcommand\circletext[1]{%
    \tikz[overlay,remember picture] 
        \node (marker-\arabic{nodemarkers}-a) at (0,1.5ex) {};%
    #1%
    \tikz[overlay,remember picture]
        \node (marker-\arabic{nodemarkers}-b) at (0,0){};%
    \tikz[overlay,remember picture,inner sep=2pt]
        \node[draw,ellipse,fit=(marker-\arabic{nodemarkers}-a.center) (marker-\arabic{nodemarkers}-b.center)] {};%
    \stepcounter{nodemarkers}%
}

% Pour les checkmark
\newcommand*\colourcheck[1]{%
  \expandafter\newcommand\csname #1check\endcsname{\textcolor{#1}{\ding{52}}}
}
\colourcheck{green}
\newcommand{\ok}[1]{\textcolor{ForestGreen}{OK \greencheck}}
\newcommand{\oktext}[1]{\textcolor{ForestGreen}{OK \greencheck}~\textcolor{ForestGreen}{#1}}

\begin{document}

%~ \tableofcontents
\newpage

%~ /data-aware-batch-scheduling/MBSS$ oarsub -p orion -l core=2,walltime=08:00:00 -r '2022-06-14 19:00:00' "bash Stats_single_workload.sh inputs/workloads/converted/test-11 inputs/clusters/rackham_450_128_32_256_4_1024.txt Fcfs 0"

\section{What's new since last meeting ?}

	\begin{enumerate}
		\item Re-writing of area filling
		\item More heatmaps fcfs with a score and all multipliers combination heatmaps
		\item Code area filling ratio and area filling omniscient
		\item pseudo code area filling with different filling of allocated area (It's: Allocated Area[choosen size][x] gets Allocated Area[choosen size][x] + Area(Ji) with Area(Ji) = cores*walltime. But I would prefer to do: cores*(delay+transfer time))
		\item code area filling with ratio and allocated area
		\item code easybf fcfs only
	\end{enumerate}
	
\section{Questions for next meeting}

	\begin{enumerate}
		\item Do I still put a data on each job ?
	\end{enumerate}

\section{Todo}
	\subsection{General}
		\begin{enumerate}
			\item revoir area filling tart dans le main en fct de planned or ratio et des nouveau noms de fichiers et retester avec les 2 cluster différents.
			\item ignore dt for constraint == 2
			\item code more easy bf and test them.
			\item are filling omnisicnet comme je fais en ce moment. Si ca marche pas essayer avec le graduellement
			\item Dire que pour omnisicent je fais l'ancienne version avec planned
			
			\item test constraint: test normal et test avec workload et cluster particulier comme écris plus bas dans la todo list pour voir si y a un impact
			\item Paye Enseirb
			\item generation de workload : aller chercher loin puis cut le fichier plus tôt pour pas avoir trop de jobs non plus.
			
			\item Check rendu rapports le 30 Aout pour les 3A
			
			\item Fusionner fcfs score et area et mettre la contrainte des tailles et les temps dechargements.
			\item ajouter backfiling a tous.
			
			\item pour area filling utiliser les stats et test avec cluster de 1/3 de gros nodes mais aussi vrai test plus gros ou je devrais pas voir de différences
			\item A la place de immedialy est ce que ca finir a plus tot sur le noeud de taille x + 1 et comparer aussi avec x + 2 ... x + n.
			\item essayer avec plus de gros noeuds pour size constraint
			\item Courbe en fixant 1 paramèrte
			\item Rajouter un max au dénominateur du stretch pour que la durée ne sois pas moins de 3 min. div max (flow parfait, 3min)
			\item Refaire heatmap
			\item Voir rapport avec A
			\item voir quantité de petits jobs
			\item Envoyer param hMETIS a Bora
			\item ajouter easy bf a fcfs avec un score si on le valide
								
			\item flow stretch courbe de points triée par walltime ou par delay au lieu de trier par job ids.
						
			\item tracer terminaison de ces jobs sur courbes usgae des nodes
					
			\item Do test with 3 new plot methods
						
			\item plot heat map of fcfs score with a parameter to 1 or 0 and the 2 others
			\item Virer 3eme multiplicateur et faire heat map pour voir des stretch
			\item prendre le stretch du flou (flow/flow cluster vide (avec transferts)) Moyenne mediane des ratio de tt les jobs ensuite.
			\item plot garder exposant sur l axe a gauche
			\item Enlever transferts sur courbes contraintes tailles et enlever les données d'entrées même si les algos les regardent
			\item mettre la rose sur la 12 pour contraintes. La 13 sert pas car trop desiquilibré
			\item plot stats ratio area et cluster
			\item Est-ce vraiment utile ?
			\item changer workload  avec bcp de petit aussi long que les gros pour prouver que strat taille a un effet mais pas visible ssur des workloads classiques. et quelques gros pour voir.
			
		
			\item voir contraintes tailles
			\item retester avec autres algos
			\item profiler mon code pour voir ce qui coute du temps
			\item Envoyer abstract ACACES avant 15 juin
			\item Faire poster ACACES
			\item Cours pour l'année prochaine
			\item Après avis favorable déposer dossier sur SIGED
			\item faire des boxplots
			\item faire option easy bf a la place de shceuler en particulier
			\item Voir thèses de Herman
			\item Talk to Hans Karlsson
			\item Lire articles Emanuel Rubensson \url{https://webmail.ens-lyon.fr/?_task=mail&_caps=pdf%3D1%2Cflash%3D0%2Ctif%3D0&_uid=4630&_mbox=INBOX&_action=show}
			\item Compare algo with different workloads. Use queue time of each job compared to a baseline. Compare queue time of each job between 2 heuristics by saying who won on each queue time.
			\item Code to compare ourselves, HEFT that would be FCFS that take into consideration transfer time for the earliest available cores
			\item Use same X scale for distributions of queue times
			\item Use same X scale workloads stats
			\item Use same X scale for some gantt charts when I want to compare them
			\item TODO du code: backfill, available node list qui contient les cores available aussi
			\item contrainte de la localié sur la taille de la données, contrainte sur les coeurs, contrainte sur les partage de données
			\item stratégie qui essaye d'utiliser moins les gros noeud pour les garder pour les gros jobs ?
			\item Lors d'une exec imprimer sur le terminal et dans le fichier de résultats les stats sur le workload et le cluster
			\item Comparer algo qui reschedule tout et algo qui schedule que quand un nouveau core est disponible et ne schedule que 1 seul job a la fois par cores.
			\item lire article bf : Utilization, Predictability, Workloads
			\item Faire plusieurs test et mettre une barre d'erreur
			\item Afficher temps de transferts sur les Gantt charts
			\item Rendre le code général en enlevant les 20 correspondant aux nombres de cores et en mettant une var globale initialisé lors de la lecture du cluster
		\end{enumerate}
	\subsection{Batsim}
		\begin{enumerate}
			\item Batsim a pas la granularité au sein d'un noeud
			\item Gérer n nodes
			\item Gérer n jobs
			\item Faire un delay aussi long que la somme du poids des données manquantes
			\item faire la maj des données du node partout, sois Dans le scheduler sois dans fit mais faut le faire!
			\item Gérér à la main les évictions
		\end{enumerate}

\end{document}
