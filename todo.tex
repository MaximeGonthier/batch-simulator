\documentclass[a4paper]{article}
\usepackage[utf8]{inputenc}
\usepackage{color}
\usepackage[dvipsnames]{xcolor}
\usepackage{graphicx}
\usepackage{amssymb}
\usepackage{amsfonts}
\usepackage{diagbox}
\usepackage{colortbl}
\usepackage{pdfpages}
\usepackage{todonotes}
\usepackage{listings}
\usepackage{amsmath}
\usepackage{caption}
\usepackage{subcaption}
\usepackage{xspace}
\usepackage{xcolor,pifont}
\usepackage{fullpage}
\usepackage{algorithm, algpseudocode}
\usepackage[hidelinks]{hyperref}
\usepackage[english]{babel}
%~ \usepackage[backend=bibtex]{biblatex}
%~ \bibliography{ref_cadre}
\algnewcommand\algorithmicforeach{\textbf{for each}}
\algdef{S}[FOR]{ForEach}[1]{\algorithmicforeach\ #1\ \algorithmicdo}
\renewcommand{\algorithmicrequire}{\textbf{Input:}}
\renewcommand{\algorithmicensure}{\textbf{Output:}}
\newtheorem{Problem}{Problem}
\newtheorem{Theorem}{Theorem}
\newtheorem{Lemma}{Lemma}
\newtheorem{Hypothesis}{Hypothesis}
\providecommand{\keywords}[1]{\textbf{\textit{Key-words:}} #1}
\newcommand{\LM}[1]{\textcolor{red}{LM:~{#1}}}
\newcommand{\ST}[1]{\textcolor{orange}{ST:~{#1}}}
\newcommand{\TODO}[1]{\textcolor{olive}{TODO:~{#1}}}
\newcommand{\Q}[1]{\textcolor{blue}{Q:~{#1}}}
\setlength{\parskip}{0.2 cm}
\title{Notes articles}
\author{Maxime GONTHIER Samuel THIBAULT Loris MARCHAL}

\newcommand{\card}[1]{\ensuremath{\left|{#1}\right|}\xspace}
\newcommand{\live}{\ensuremath{L}\xspace}
\newcommand{\evict}{\ensuremath{\mathcal{V}}\xspace}
\newcommand{\dataset}{\ensuremath{\mathbb{D}}\xspace}
\newcommand{\taskset}{\ensuremath{\mathbb{T}}\xspace}
\newcommand{\packageset}{\ensuremath{\mathbb{P}}\xspace}
\newcommand{\inputs}{\ensuremath{\mathcal{D}}\xspace}
\newcommand{\optpb}{\textsc{Min\-Loads\-For\-Tasks\-Sharing\-Data}\xspace}
\newcommand{\nbloads}{\ensuremath{\mathit{\mathit{\#Loads}}}\xspace}
\newcommand{\MIN}{\ensuremath{\mathit{MIN}}\xspace}
\newcommand{\starpu}{\textsc{StarPU}\xspace}
\newcommand{\nbPU}{\ensuremath{\vert PU \vert}\xspace}

\usepackage{tikz}
\usetikzlibrary{fit, shapes.geometric, patterns}

\makeatletter\tikzset{hatch distance/.store in=\hatchdistance,hatch distance=5pt,hatch thickness/.store in=\hatchthickness,hatch thickness=5pt}\pgfdeclarepatternformonly[\hatchdistance,\hatchthickness]{north east hatch}{\pgfqpoint{-1pt}{-1pt}}{\pgfqpoint{\hatchdistance}{\hatchdistance}}{\pgfpoint{\hatchdistance-1pt}{\hatchdistance-1pt}}{\pgfsetcolor{\tikz@pattern@color}\pgfsetlinewidth{\hatchthickness}\pgfpathmoveto{\pgfqpoint{0pt}{0pt}}\pgfpathlineto{\pgfqpoint{\hatchdistance}{\hatchdistance}}\pgfusepath{stroke}}\makeatother\usetikzlibrary{calc,shadings,patterns,tikzmark}\newcommand\HatchedCell[5][0pt]{\begin{tikzpicture}[overlay,remember picture]\path ($(pic cs:#2)!0.5!(pic cs:#3)$)coordinate(aux1)(pic cs:#4)coordinate(aux2);\fill[#5]($(aux1)+(-0.23*0.075\textwidth,1.9ex)$)rectangle($(aux1 |- aux2)+(0.23*0.075\textwidth,-#1*\baselineskip-.8ex)$);\end{tikzpicture}}

\newcounter{nodemarkers}
\newcommand\circletext[1]{%
    \tikz[overlay,remember picture] 
        \node (marker-\arabic{nodemarkers}-a) at (0,1.5ex) {};%
    #1%
    \tikz[overlay,remember picture]
        \node (marker-\arabic{nodemarkers}-b) at (0,0){};%
    \tikz[overlay,remember picture,inner sep=2pt]
        \node[draw,ellipse,fit=(marker-\arabic{nodemarkers}-a.center) (marker-\arabic{nodemarkers}-b.center)] {};%
    \stepcounter{nodemarkers}%
}

% Pour les checkmark
\newcommand*\colourcheck[1]{%
  \expandafter\newcommand\csname #1check\endcsname{\textcolor{#1}{\ding{52}}}
}
\colourcheck{green}
\newcommand{\ok}[1]{\textcolor{ForestGreen}{OK \greencheck}}
\newcommand{\oktext}[1]{\textcolor{ForestGreen}{OK \greencheck}~\textcolor{ForestGreen}{#1}}

\begin{document}

%~ \tableofcontents
\newpage

%~ /data-aware-batch-scheduling/MBSS$ oarsub -p orion -l core=2,walltime=08:00:00 -r '2022-06-14 19:00:00' "bash Stats_single_workload.sh inputs/workloads/converted/test-11 inputs/clusters/rackham_450_128_32_256_4_1024.txt Fcfs 0"

\section{What's new since last meeting ?}

	\begin{enumerate}
		\item Re-writing of area filling
		\item More heatmaps fcfs with a score and all multipliers combination heatmaps
		\item Code area filling ratio and area filling omniscient
		\item pseudo code area filling with different filling of allocated area (It's: Allocated Area[choosen size][x] gets Allocated Area[choosen size][x] + Area(Ji) with Area(Ji) = cores*walltime. But I would prefer to do: cores*(delay+transfer time))
		\item code area filling with ratio and allocated area
		\item code easybf fcfs only
		\item are filling omnisicnet comme je fais en ce moment. Si ca marche pas essayer avec le graduellement
		\item dans area filling: A la place de immedialy est ce que ca finir a plus tot sur le noeud de taille x + 1 et comparer aussi avec x + 2 ... x + n.
		\item Size constraint: sorting by size is bad if normal cluster
		\item ignore dt for constraint == 2
		\item revoir area filling tart dans le main en fct de planned or ratio et des nouveau noms de fichiers et retester avec les 2 cluster différents.
		\item plot with nb of upgraded jobs
		\item plot avec backfill. Ameliore fcfs mais fcfs score est meilleur sans. La diff e transfert est grande entre fcfs score et fcfs score bf
		\item Why are filling so bad ? Even in omniscient ? I corrected a bit but it's still worse. Also there is a difference between true allocated area and the one I compute in the schedule (because the scheduele change and I only update allocated area when a job start. Else I have a temp one just for the current schedule).
		\item Try in area filling to reduce allocated area by the diff between walltime and real delay if the job finished before it's walltime
		\item Comment one paper of vincenc
		\item Mail au gars de data locality sur numa: Jannis Klinkenberg
		\item dossier inscription administrative
		\item attestation sur siged de acaces et IPDPS
		\item plot the number of jobs where the queue time is superior to 25000 for example
		\item tirer quelques conclusions dans le document principal
		\item plot 2 contraintes ensembles
		\item code pour tracer terminaison des jobs -2 sur courbes usage du cluster: mais toujours pas full. Je devrais prendre bcp plus loin dans le futur ?
		\item tester en faisant varier le multiplicateur de fcfs score area filling: je n'observe pas d'améliorations
		\item Afficher temps de transferts sur les Gantt charts
	\end{enumerate}
	
\section{Questions for next meeting}
	\begin{enumerate}
		\item HEFT is like fcfs with a score x1 x0 x0 x0 I guess ? Oui
		\item On a eu 2 reviews sur fgcs
		\item Do I still put a data on each job ?
		\item Utiliser attestation volontariat IPDPS pour formations ? 20h ? Pareil pour ACACES avec 20h de cours ?
		\item Still, no corelation between node size and job length. Should I add one ?
	\end{enumerate}
	
\item Pour le workload: ignorer jobs cancelled. Ne pas ignorer job failed ou timeout. si jobs multi node les diviser. mettre un overhead de 2*1min aux jobs et de 1*2min aux failed. Par contre les cancelled je les ignore complètement.
	
\faire varier les ratio de chaque taille voir si a un moment l'agressivité est detrimentale
\ajouter bf aux autres strates de area filling et backfill big nodes
\prendre 1 mois entier et couper pour voir si on tout
\demander a carl a quel point c'ets utilisé. Y a til de l'overhead ? Ce qui expliquerai l'usage du cluster
\prendre en compte les jobs cancelled et timeout pour le schedule , les error a voir pour l'usage de la plateforme
y a t il un overhead avant apres la terminaison d n job ? Le simuler ?
regarder qui sont les jobs delaier de plus de 25000

\section{Informations for the article}
	\begin{enumerate}
		\item Use same X scale for some gantt charts when I want to compare them: to do that add a job of duration 0 at the maxtime of the heuristic you are showing
		\item Faire plusieurs test et mettre une barre d'erreur (ce qui varie c'est la taille des fichiers d'entrées lors de la génération du workload). For this I did Generate\_workload\_from\_rackham\_for\_variance.sh to generate different workloads (10 is enough or I go up to 30 ? Might take a while if I do 30 times the same test. Or maybe just once to show the varaince is small ?) and then I try them all and show the variance.
		\item Use same X scale for workloads stats when I want to compare them
		\item Code to compare ourselves, HEFT that would be FCFS that take into consideration transfer time for the earliest available cores: It is just fcfs with a score x1 x0 x0 x0
		\item test Fcfswithascoreareafillingomniscientx qui varie: pas de meilleurs résultats
		\item cluster usage in real: https://www.uppmax.uu.se/resources/system-usage/rackham/
	 \end{enumerate}
	
\section{Todo}
	\subsection{General}
		\begin{enumerate}
			\item se renseigner payement enseirb semestre 1: (demander a sandrine truffier ?)
			\item remboursement loris frejus papier
			\item payer ré inscription
			\item Voir mission Dallas: demander a denis
			\item mail calendrier mes autres cours enseirbs: attendre réponse
			\item Répondre a Bora
			\item transmettre dossier siged  et payer une fois tous les avis saisies (5/6 ou 6/6)
			\item Check rendu rapports le 30 Aout pour les 3A
			\item Date redu rapport des 2A ?
		\end{enumerate}
	\subsection{Uppsala}
		\begin{enumerate}
			\item test autres workloads pour les 2 contraintes
			\item voir pk ca crash avec bcp de jobs
			\item mettre a jour planned area pour fcfs score area filling quand j'upgrade des jobs ?
			\item generation de workload : aller chercher loin puis cut le fichier plus tôt pour pas avoir trop de jobs non plus ?
			\item Ne pas oublier que j'ai les courbes VS sur flow stretch et queue times aussi que je peux tracer						
			\item Rendre le code général (20 pour cores par node et 3 pour nb tailles de noeuds différents)
		\end{enumerate}
\end{document}
